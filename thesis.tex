\documentclass[a4paper, 10pt]{article}
\usepackage[utf8]{inputenc}
\usepackage{xspace} % Needed for \xspace
\usepackage{hyperref} % Needed for \href
\usepackage{graphicx} % Needed for \includegraphics
\usepackage{float} % Needed for figures with the H placement specifier

% Utility macros
\newcommand{\atweb}{\textbf{@Web}\xspace}
\newcommand{\code}[1]{\texttt{#1}}
\newcommand{\img}[3]{
  \begin{figure}[H]
    \begin{center}
      \includegraphics[width=#3]{img/#1}
      \caption{#2}
    \end{center}
  \end{figure}
}

% Formatting
\setlength{\parskip}{1em}

\begin{document}

In this thesis we study different techniques for expressing and verifying
integrity constraints over data stored in ontologies built using Semantic Web
technologies.

This work is motivated by data validation requirements in the @Web platform, a
semantic web application that allows domain experts to extract data in
scientific documents, and researchers to explore and query those data via a
graphical user interface. The extracted data is stored in a publicly
accessible RDF graph following a predefined OWL ontology, and shared with the
research community.

Given the error-prone nature of the data extraction process, a set of
integrity constraints has been identified that all extracted data must
fulfill. It is desired to validate these constraints automatically and report
any validation errors to the domain expert during the data extraction process.

To this end, we first survey the current W3C recommendations for querying,
describing and constraining the contents of RDF graphs and the available tools
implementing these recommendations. We decide to focus our analysis on SPARQL,
Shape Expressions and SHACL. We then implement a set of test constraints using
each of the available tools and compare them according to expressiveness,
verbosity, readability, running times, etc. Finally, we identify the tool that
best suits our concrete needs and proceed to implement the constraint
validation features in @Web.

Our analysis shows that plain SPARQL queries yield the best running times
amongst the technologies considered. We also observe that certain constraints
expressed in both Shape Expressions and SHACL require nesting SPARQL queries
that are comparable in length to stand-alone SPARQL queries implementing those
same constraints, thus defeating the purpose of an alternate constraint
language. To conclude, we propose modifications to both the Shape Expressions
and SHACL languages that would allow all our use cases.

\textbf{Keywords}: Semantic Web, RDF, OWL, SPARQL, Shape Expressions, SHACL,
Constraints, @Web.

\end{document}

