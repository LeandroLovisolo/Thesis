\documentclass[a4paper, 10pt]{article}
\usepackage[margin=1.5in]{geometry}
\usepackage[utf8]{inputenc}
\usepackage{amssymb}
\usepackage{hyperref} % Needed for \href
\usepackage{xspace} % Needed for \xspace
\usepackage{graphicx} % Needed for \includegraphics
\usepackage{float} % Needed for figures with the H placement specifier

% Utility macros
\newcommand{\header}[1]{
  \begin{center}
    \textbf{\uppercase{#1}}
  \end{center}
  \vspace{-1em}
}
\newcommand{\fnhref}[2]{\href{#2}{#1}\footnote{\url{#2}}}
\newcommand{\ifnhref}[2]{\textit{\fnhref{#1}{#2}}}
\newcommand{\code}[1]{\texttt{#1}}
\newcommand{\img}[3]{
  \begin{figure}[H]
    \begin{center}
      \includegraphics[width=#3]{img/#1}
      \caption{#2}
    \end{center}
  \end{figure}
}

% Paragraph formatting
\setlength{\parskip}{1em}

% Prevent hyphenation
\tolerance=1
\emergencystretch=\maxdimen
\hyphenpenalty=10000
\hbadness=10000

% Bullets in nested \itemize environments
\renewcommand{\labelitemii}{$\circ$}
\renewcommand{\labelitemiii}{---}

%%%%%%%%%%%%%%%%%%%%%%%%%%%%%%%%%%%%%%%%%%%%%%%%%%%%%%%%%%%%%%%%%%%%%%%%%%%%%%%%

\begin{document}

\header{Summary}

In this thesis we study different techniques for expressing and verifying
integrity constraints over data stored in ontologies built using
\ifnhref{Semantic Web}{https://en.wikipedia.org/wiki/Semantic\_Web}
technologies.

This work is motivated by data validation requirements in the \ifnhref{@Web
platform}{https://www6.inra.fr/cati-icat-atweb}, a Semantic Web application
that allows domain experts to annotate experimental data in scientific
documents, and researchers to explore and query those annotations via a
graphical user interface. The annotations are stored in a publicly accessible
\ifnhref{RDF}{http://www.w3.org/TR/rdf11-primer/} graph using a vocabulary
predefined in an \ifnhref{OWL}{http://www.w3.org/TR/owl-primer/} ontology, and
shared with the research community.

Given the error-prone nature of the data annotation process, a set of integrity
constraints has been identified that all annotated data must fulfill. It is
desired to validate these constraints automatically and report any validation
errors to the domain expert during the data annotation process.

To this end, we first survey the current W3C recommendations for constraining
the contents of RDF graphs, and the available tools implementing these
recommendations. We decide to focus our analysis on
\ifnhref{SPARQL}{http://www.w3.org/TR/sparql11-query/}, \ifnhref{Shape
Expressions}{http://www.w3.org/2013/ShEx/Primer} and
\ifnhref{SHACL}{http://www.w3.org/TR/shacl/}. We then implement a set of test
constraints using each of the available tools and compare them according to
expressiveness, verbosity, readability, etc.

Our analysis shows that none of the libraries implementing the Shape
Expressions recommendation fully support all our use cases. We also observe
that certain constraints expressed in SHACL require nesting SPARQL queries that
are comparable in length to stand-alone SPARQL queries implementing those same
constraints, thus defeating the purpose of an alternate constraint language.

We finish our analysis by comparing the running times of a set of test
constraints implemented as SPARQL queries against different triple stores
supported by \ifnhref{Jena}{https://jena.apache.org/}, a Java library for
building Semantic Web applications which is already used in @Web for other
purposes.

With this information, we choose SPARQL as the technology for expressing
integrity constraints in @Web, we adopt the triple store that yields the best
running times for our test constraints, and we proceed to implement the
constraint verification features in the @Web platform.

To conclude, we suggest modifications to both the Shape Expressions and SHACL
languages that would allow all our use cases.

\textbf{Keywords}: Semantic Web, RDF, OWL, SPARQL, Shape Expressions, SHACL,
Constraints, @Web.

%%%%%%%%%%%%%%%%%%%%%%%%%%%%%%%%%%%%%%%%%%%%%%%%%%%%%%%%%%%%%%%%%%%%%%%%%%%%%%%%

\newpage

\header{Thesis outline}

\begin{itemize}
  \item Summary

  \item Introduction

  \textit{Problem statement, introduction to the @Web platform, examples of
  actual constraints we would like to verify.}

  \item Overview of Semantic Web technologies

  \begin{itemize}
    \item RDF
    \item RDFS
    \item OWL
  \end{itemize}

  \item Survey of current W3C recommendations

  \textit{Brief introduction to the recommendations evaluated.}

  \begin{itemize}
    \item SPARQL
    \item Shape Expressions
    \item SHACL
  \end{itemize}

  \item Evaluation of W3C recommendations

  \textit{Description of the evaluation methodology, criteria used, etc.}

  \begin{itemize}
    \item Constraints used for the evaluation of technologies

    \item Constraint implementations

    \begin{itemize}
      \item SPARQL
      \item Shape Expressions
      \item SHACL
    \end{itemize}

    \item Discussion

    \item Conclusions
  \end{itemize}

  \item Evaluation of SPARQL

  \begin{itemize}
    \item SPARQL queries used during experimentation
    \item Triple store implementations evaluated
    \item Experiments
    \item Results
    \item Discussion
    \item Conclusions
  \end{itemize}

  \item Implementation

  \begin{itemize}
    \item Functional specification
    \item Discussion
  \end{itemize}

  \item Conclusions

  \item Suggested modifications to the Shape Expressions and SHACL languages

  \item Future work
\end{itemize}

\end{document}
